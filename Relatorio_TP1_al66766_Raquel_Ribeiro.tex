%----------------------------------------------------------------------------------------
%	PACKAGES AND OTHER DOCUMENT CONFIGURATIONS
%----------------------------------------------------------------------------------------
\documentclass[12pt]{article}
\usepackage[UTF8]{inputenc}
\usepackage[portuguese]{babel}
\usepackage[scaled]{helvet}
\usepackage[T1]{fontenc}
\usepackage{amsmath}
\usepackage{amsthm}
\usepackage{amsfonts}
\usepackage{tabularx}
\usepackage{amssymb}
\usepackage{makeidx}
\usepackage{float}
\usepackage{graphicx}
\usepackage{comment}
\usepackage{hyperref}
\usepackage[justification=centering]{caption}
\DeclareMathOperator{\sech}{sech}
\DeclareMathOperator{\arcosh}{arcosh}
\usepackage[left=2cm,right=2cm,top=2cm,bottom=2cm]{geometry}
\usepackage{pdfpages}
\author{dxcccii}

\renewcommand\familydefault{\sfdefault}

\begin{document}

%----------------------------------------------------------------------------------------
%	TITLE PAGE
%----------------------------------------------------------------------------------------

\begin{titlepage} % Suppresses displaying the page number on the title page and the subsequent page counts as page 1
	\newcommand{\HRule}{\rule{\linewidth}{0.5mm}} % Defines a new command for horizontal lines, change thickness here
	
	\center % Centre everything on the page
	
%------------------------------------------------
%Headings
%------------------------------------------------
	
	\textsc{\LARGE UNIVERSIDADE TRAS OS MONTES E ALTO DOURO}\\[1.5cm] % Main heading such as the name of your university/college
	
	\textsc{\Large engenharia informática}\\[0.5cm] % Major heading such as course name
	
	\textsc{\large SISTEMAS DISTRIBUIDOS}\\[0.5cm] % Minor heading such as course title
	
%------------------------------------------------
%Title	
%------------------------------------------------
	
	\HRule\\[0.4cm]
	
	{\huge\bfseries TRABALHO PRACTICO 1\\ 
	\hfill \\
	SERVIDOR DE GESTÃO DE SERVIÇOS DE MOBILIDADE}\\[0.4cm] % Title of your document
	
	\HRule\\[1.5cm]
	
%------------------------------------------------
%Author(s)	
%------------------------------------------------
	
	\begin{minipage}{0.4\textwidth}
		\begin{flushleft}
			{\Large
			\textsc{autor}}\\
			 {\large Raquel Ribeiro\\ al66766@utad.eu\\Turma Pratica 1} % Your name
		\end{flushleft}
	\end{minipage}
	~
	\begin{minipage}{0.4\textwidth}
		\begin{flushright}
			\Large
			\textsc{docentes}\\
			{\large Prof. Hugo Paredes\\ Prof. Tiago Pinto }
		\end{flushright}
	\end{minipage}
%------------------------------------------------
%Date & illustration
%------------------------------------------------

\begin{figure}[H]
    \centering
    \includegraphics[width=12cm]{ilustracaocapa}
\end{figure}
	
	\vfill\vfill\vfill % Position the date 3/4 down the remaining page
	
	{\large\today} % Date, change the \today to a set date if you want to be precise
	
%------------------------------------------------
%Logo
%------------------------------------------------
	
	%\vfill\vfill
	%\includegraphics[width=0.2\textwidth]{placeholder.jpg}\\[1cm] % Include a department/university logo - this will require the graphicx package
	 
	%----------------------------------------------------------------------------------------
	
	\vfill % Push the date up 1/4 of the remaining page
	
\end{titlepage}

%------------------------------------------------
% INDEX & TABLE OF CONTENTS


%\textsc{\tableofcontents}

%\newpage


% INTRODUCTION
%------------------------------------------------
\section{// PROTOCOLO DE COMUNICAÇÃO}
O protocolo de comunicação entre o cliente e o servidor é definido por uma série de comandos trocados através de uma conexão TCP.\\

\noindent As principais mensagens e os seus propósitos do lado do cliente sao:
\begin{itemize}
  \item CONNECT: Inicia a ligação do cliente com o servidor.
  \item CLIENT ID:[id]: O cliente envia esta mensagem para se identificar ao servidor. so-called bullet.
  \item REQUEST TASK CLIENT ID:[id]: O cliente solicita uma nova tarefa.
  \item TASK COMPLETED: [descrição da tarefa]: O cliente informa o servidor que uma tarefa foi concluída.
  \item REQUEST SERVICE CLIENT ID:[id]: O cliente solicita um serviço.
  \item SAIR: O cliente informa o servidor que deseja desconectar-se.
\end{itemize}

\noindent No lado do servidor, as mensagens de resposta ao cliente sao:  
\begin{itemize}
  \item 100 OK: Ligação estabelecida.
  \item 400 BYE: O servidor reconhece o pedido de desconexão do cliente.
  \item 500 ERROR: [mensagem de erro]: O servidor encontrou um erro.
  \item ID CONFIRMED:[id]: O servidor confirma o ID do cliente.
  \item SERVICE ALLOCATED:[serviço]: O servidor atribui um serviço ao cliente.
  \item TASK ALLOCATED:[descrição da tarefa]: O servidor atribui uma tarefa ao cliente.
  \item NO SERVICE AVAILABLE: Não há serviços disponíveis para o cliente.
  \item NO TASKS AVAILABLE: Não há tarefas disponíveis.
  \item TASK MARKED COMPLETED: Tarefa marcada como concluída.
\end{itemize}



\end{document}